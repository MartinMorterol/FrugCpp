\documentclass[a4paper, 11pt]{article}
\usepackage{listings}
\usepackage{amsmath,amsfonts,amssymb}
\usepackage{graphicx}
\usepackage{hyperref}
\usepackage{float}
\usepackage{enumitem}
\usepackage{placeins}
\usepackage{multirow,tabularx}
\usepackage{algorithm}
\usepackage{algorithmic}
\begin{document}

	\title{Détails techniques}
	\maketitle
	\tableofcontents
 	
\section{classe spécialisation}

\begin{itemize}
	\item Si il existe une spécialisation (complète ou partielle) qui match avec l'utilisation de la classe, cette spécialisation sera utilisée. 
	\item Si l'utilisation de la classe à moins de type que dans la déclaration de la classe, les types par défaut seront affecté.
	\begin{lstlisting}
template <class T, class U = int >
struct demo
{
    constexpr static int value = 0;
};

template<class T>
struct demo <T,int>
{
    constexpr static int value = 1;
};

template<class T>
struct demo <T,void>
{
    constexpr static int value = 2;
};

int main() {
    using namespace std;
    cout << demo<std::string>::value << endl;
    cout << demo<int,int>::value << endl;
    cout << demo<int,void>::value << endl;
    cout << demo<int,std::string>::value << endl;

}
	\end{lstlisting}
\end{itemize}
Va afficher : "1 1 2 0". Les réponses 2 et 3 sont sans surprise l'utilisation de la spécialisation. La réponse 4 est le cas ou aucune spécification ne match et la réponse 1 montre que lorsqu'on à qu'un seul argument template, les valeurs par défauts sont appliquées puis la sélection de la classe la plus spécialisé se fait. Ainsi dans cet exemple tout appel à "demo" avec un paramètre matchera avec la spécification <T,int>

\section{SFINAE}

Le principe de SFINAE : "Substitution failure is not an error".
Lorsque le compilateur résous un type ou un appel de fonction il veut au final avoir exactement un type/fonction a appeler (une fois les regles de priorité appliqué s'il y en a). S'il y en plus il ne saura pas quoi choisir, et s'il n'y en a pas il ne pourra rien faire. \\
Ce que SFINAE dit c'est "Si tu trouves quelque chose qui match, regarde si la signature + type de retour à un sens en terme de type, si c'est pas le cas, c'est pas grave, passe au suivant". 
\\Avec un exemple sur une fonction (ce n'est pas jsute de la surcharge de fonction car toutes ces fonctions peuvent être appelé avec un T)

On notera dans le code la présence de "typename" lorsqu'un utilise un type d'une classe template "T::monType". Comme la syntaxe "T::qqch" peut aussi bien etre un type qu'un attribut statique, le compilateur nous demande de préciser avec "typename" lorsqu'il s'agit d'un type.

\begin{lstlisting}
template <class T>
typename T::value_type fonction_sfinae (const T& t){
    std::cout << "Le type T::value_type a un sens " << std::endl;
    return typename T::value_type(); // je supose qu'il est default
    // constructible mais c'est juste pour l'exemple.
}

template <class T>
void fonction_sfinae(T t, typename T::bibi d = 1) {
     std::cout << "Le type T::bibi a un sens " << std::endl;
}

template <class T, class U = typename T::boo >
void fonction_sfinae(T t) {
     std::cout << "Le type T::boo a un sens " << std::endl;
}

struct booclass {
    using boo = int;
};

struct bibiclass {
    using bibi = int;
} ;

struct drame {
    using bibi = int;
    using boo = int;
} ;
int main() {
    using namespace std;

    fonction_sfinae(vector<int>());
    fonction_sfinae(booclass ());
    fonction_sfinae(bibiclass ());
    /*fonction_sfinae(drame ());
    |39|error: call of overloaded ‘fonction_sfinae(drame)’ 
    is ambiguous|
    |39|note: candidates are:|
    |12|note: void fonction_sfinae(T, typename T::bibi) 
    [with T = drame; typename T::bibi = int]|
    |17|note: void fonction_sfinae(T) [with T = drame; U = int]|*/
    // fonction_sfinae(5); |44|error: no matching function for
     call to ‘fonction_sfinae(int)’|
}
\end{lstlisting}
L'affichage sans surprise : \\
Le type T::value\_type a un sens \\
Le type T::boo a un sens \\
Le type T::bibi a un sens \\
Pour le cas de "drame" on a comme le compilateur nous l'indique, deux fonctions qui match. Il ne peut pas choisir.
Et le cas de l'entier est le cas ou personne ne match.

\section{decltype, declval, enbale\_if et autre joyeuseté} 

decltype : permet de retourner le type d'une expression, ça saute pas forcement au yeux comme ça, mais c'est génial.
\begin{lstlisting}
	int a = 4 ;
	decltype(a) b = 6; // a est un int
	decltype(monObjet.maMethode()) k; 

	template<
            class T,
            class U = decltype(T().begin()),
            class W = decltype(T().end())
        >
     .... 
\end{lstlisting}
Le fonctionnement est assez proche de auto, mais les règles ne sont pas exactement les "const" et les référence. ( google "auto vs decltype vs decltype(auto)" ). \\


declval : répond au probleme suivant : Si on veut connaître le type de retour d'un méthode donc doit faire un decltype sur l'appel de la méthode. Mais comment fait on lorsque le type n'est pas default constructible ? 
class W = decltype(T().end()) // marche si T() est valide
class W = decltype(std::declval<T>().end()) // marche.
\\
enbale\_if : enable\_if<false>::type n'existe pas et enable\_if<true>::type existe (et ça vaut void)
\\ 
is\_same : is\_same<T,U>::value vaut vrais si T==U et faux si non.

A noter : 
\begin{lstlisting}
struct A { 
	constexpr static bool value = true;
};
équivaut à 
struct A : std::true_type{}
\end{lstlisting}

\section{Application, affichage des conteneurs de la STL}
Avec ça la spécialisation, SFINEA et ce qu'on vient de voir on peut crée notre premier type trait. 

\begin{lstlisting}
template <class T, class U = void, class V = void>
struct is_container : std::false_type {};

template <class T>
struct is_container<   T,
                	 	decltype(std::declval<T>().begin()),
               			decltype(std::declval<T>().end()
            >  : std::true_type {};
\end{lstlisting}
Et la ... ça ne marche pas :D. Pourquoi ?
\begin{itemize}
	\item  Si T n'est pas un conteneur, la spécialisation n'est pas pris en compte car elle n'est pas valide.
	\item  Si T est un conteneur on va cherche un match avec T void void. La spécialisation échoue car elle spécialise T T::iterator  T::iterator ou quelque chose dans se gout la.
\end{itemize}
On veut du void ! Dans se cas : 
\begin{lstlisting}
template<class T>
struct void_if_valide {
    using type = void;
};

template <class T, class U = void, class V = void>
struct is_container : std::false_type {};

template <class T>
struct is_container<   T,
                typename void_if_valide<decltype(std::declval<T>().begin())>::type,
                typename void_if_valide<decltype(std::declval<T>().end())>::type
            >  : std::true_type {};
\end{lstlisting}

Avec ça on peut (relativement) simplement afficher tous les conteneurs de la STL (il faudra faire operator << pour paire si on veut aussi afficher les map). On prendra soins que notre opérateur ne match pas les string et les wstring pour lesquels l'operateur << existe déjà.
\begin{lstlisting}
template<class T>
struct void_if_valide {
    using type = void;
};

template <class T, class U = void, class V = void>
struct is_container : std::false_type {};

template <class T>
struct is_container<   T,
                typename void_if_valide<decltype(std::declval<T>().begin())>::type,
                typename void_if_valide<decltype(std::declval<T>().end())>::type
            >  : std::true_type {};

template <class T, class U = void>
struct is_string : std::false_type {};

template <class T>
struct is_string<   T,
                typename void_if_valide<typename std::enable_if<
                                                                std::is_same<T,std::wstring>::value ||
                                                                std::is_same<T,std::string>::value
                                                                >::type
                                        >::type
            >  : std::true_type {};



template<
        class T,
        class V = typename std::enable_if< is_container<T>::value >::type,
        class U = typename std::enable_if< !is_string<T>::value >::type
      >
std::ostream& operator<< (std::ostream& out, const T& container)
{
    for (const auto& val : container)
    {
       // out << "\t" << val << "\n" ;
       out << val << " " ;
    }
    return out;
}
\end{lstlisting}



\end{document}


