\documentclass{beamer}
\usepackage{listings}
\usepackage[utf8]{inputenc}  
\usepackage[french]{babel}
\usepackage[T1]{fontenc} 
\usepackage{listings}



\graphicspath{images/}
\usetheme{Luebeck}
\usecolortheme{crane}
\definecolor{OliveGreen}{rgb}{0.1,0.6,0.1}
\lstset{language=C++,
                keywordstyle=\color{blue},
                stringstyle=\color{red},
                commentstyle=\color{OliveGreen},
                morecomment=[l][\color{magenta}]{\#},
                 basicstyle=\footnotesize\ttfamily,
                 morekeywords={decltype }
}




% un test pour pouvoir barer qqch dans le titre de beamer
\newlength{\textlarg}
\newcommand{\barre}[1]{%
   \settowidth{\textlarg}{#1}
   #1\hspace{-\textlarg}\rule[0.5ex]{\textlarg}{0.5pt}}

% la gestion des sommaires
\AtBeginSection[]
{
 \begin{frame}
  \frametitle{Sommaire}
  \tableofcontents[currentsection, hideothersubsections]
  \end{frame} 
}


\AtBeginSubsection[]
{
 \begin{frame}
  \frametitle{Sommaire}
  \tableofcontents[  
    sectionstyle=show/shaded,
    subsectionstyle=show/shaded/hide
    ]
  \end{frame} 
}

% l'inbrication de itemize
\setbeamertemplate{itemize subitem}{\tiny\raise1.5pt\hbox{\donotcoloroutermaths$\blacktriangleright$}}

% le pied de page en 3 parties avec des n° de ligne
\setbeamertemplate{footline}{
\leavevmode%
\hbox{\hspace*{-0.06cm}
\begin{beamercolorbox}[wd=.2\paperwidth,ht=2.25ex,dp=1ex,center]{author in head/foot}%
	\usebeamerfont{author in head/foot}\insertshortauthor%~~(\insertshortinstitute)
\end{beamercolorbox}%
\begin{beamercolorbox}[wd=.6\paperwidth,ht=2.25ex,dp=1ex,center]{section in head/foot}%
	\usebeamerfont{section in head/foot}\insertshorttitle
\end{beamercolorbox}%
\begin{beamercolorbox}[wd=.2\paperwidth,ht=2.25ex,dp=1ex,right]{section in head/foot}%
	\usebeamerfont{section in head/foot}
	\insertframenumber{} / \inserttotalframenumber\hspace*{2ex}
\end{beamercolorbox}}%
\vskip0pt%
}




\title{Déduction des types des paramètres}
\author{Martin~Morterol}
\date{\today}


\begin{document}

%le titre
\begin{frame}[plain]
\titlepage
\end{frame}

\section{Pourquoi ?}
\begin{frame}
	Cas d'utilisations : 
	\begin{itemize}
		\item Le premier est celui annoncé dans le résumé du talk.
		\item[] Stocker une fonction passée en tant que template dans une \lstinline{std::function}.
		\item Le deuxième est faire un message d'erreur propre lors d'un mauvais passage de fonction.
	\end{itemize}
\end{frame}
\begin{frame}
	Contexte : Un petit jeu en 2D.
    \begin{itemize}
    	\item Une carte est un conteneur (comme vector, map...).
    	\item On peut donc l'utiliser dans un for range.
    \end{itemize}
    Problème : 
    \begin{itemize}
    	\item Une carte contient des \lstinline{std::reference_wrappeur<Case>}.
    	\item Je n'ai pas envie que l'utilisateur connaisse ma structure interne ni qu'il en dépende.
    	\item Je n'ai pas envie d’écrire .get() à chaque fois. 
    	\item J'ai envie de me faire plaisir ça sert à ça les projets perso :)
    \end{itemize}
    Solution : Je crée mon iterateur maison que j'appelle FoncteurIterator.
  
    
\end{frame}
\begin{frame}[containsverbatim]

	Exemple : \href{run:../code_demo/iterator.cpp}{iterator.cpp}		
		
\end{frame}
\begin{frame}[containsverbatim]
	Problème : 
	\begin{itemize}
		\item Pour utiliser un \lstinline{std::function<>} J'ai besoin du type de retour du foncteur et du type du paramètre. 
		\item Je ne voulais pas juste utiliser "FoncteurTemplate foncteur" (bien que ça soit le seule solution valide en C++14) pour me forcer à utiliser des choses que je ne connaissait pas.
	\end{itemize}
	Solution :
	\begin{itemize}
		\item Solution Ok mais moche : 
	\begin{lstlisting}
FoncteurIterator< IteratorTemplates,
    FoncteurTemplate,typeRetour, typeParam>
	\end{lstlisting}		
		\item Trouver un moyen de déduire typeRetour et typeParam directement de FoncteurTemplate. 
	\end{itemize}
\end{frame}

\begin{frame}
	Pour la deuxième application :  
\\
		Problème : 
	\begin{itemize}
		\item Les erreurs des méthodes attendant un "callable" sont vites moches.
		\item Exemple avec \lstinline{std::find_if}.
	\end{itemize}
	Solution :
	\begin{itemize}
		\item Encapsuler \lstinline{std::find_if}.
		\item Trouver un moyen de faire de \lstinline{static_assert} sur le type de retour et les paramètres. 
	\end{itemize}
	
	Les messages, dans ce cas, ne sont pas horrible mais l’amélioration des messages d'erreur est toujours un plus, surtout en méta-programmation.
\end{frame}

\section{Prérequis}
%\begin{frame}
%	Je vais présenter les techniques que j'ai utilisé puis vous présenter la solution en elle même. Il n'y a pas de grosse difficulté mais ça peut faire beaucoup de chose nouvelles pour ceux qui n'ont jamais fait de méta-programmation. 
%\end{frame}

\subsection{La spécialisation}

\begin{frame}[containsverbatim]
	La spécialisation (partielle ou non) est le fait de, pour une structure, spécifier une partie (ou tous) des types. Dans se cas le compilateur choisir la structure la plus spécialiser.\\
	
	\begin{lstlisting}
template< class T > struct is_pointer_helper    
                     : std::false_type {};
template< class T > struct is_pointer_helper<T*> 
                     : std::true_type {};
	\end{lstlisting}
	source du code : \href{http://en.cppreference.com/w/cpp/types/is_pointer}{cppreference}
\end{frame}

\begin{frame}[containsverbatim]
	Si la structure à des types par défaut ils sont déduits avant de regarder quelle structure est la plus spécialisée.
	\begin{lstlisting}
template <class T, class U = int >
struct demo { constexpr static int value = 0 ;};

template<class T>
struct demo <T,int> {constexpr static int value = 1;};

template<class T>
struct demo <T,void> {constexpr static int value = 2;};

cout << demo<std::string>::value << endl;     // 1
cout << demo<int,int>::value << endl;         // 1
cout << demo<int,void>::value << endl;        // 2
cout << demo<int,std::string>::value << endl; // 0
	\end{lstlisting}
	
\end{frame}

\subsection{SFINAE}
\begin{frame}[containsverbatim]

	\begin{itemize}
		\item Substitution failure is not an error : Si on arrive pas à résoudre la déclaration, on l'ignore.
		\begin{itemize}
			\item Pour les structures : Dans les type template et les types de spécialisation.
			\item Pour les fonctions : Dans les types template, les arguments, et le type de retour.
		\end{itemize}
		\item Exemple : \href{run:../code_demo/sfinae.cpp}{sfinae.cpp}
	\end{itemize}

	
\end{frame}

\subsection{Outils utiles}

\begin{frame}[containsverbatim]
decltype : permet de retourner le type d'une expression, ça saute pas forcement au yeux comme ça, mais c'est génial.

\begin{lstlisting}
	int a = 4 ;
	decltype(a) b = 6; // a est un int
	decltype(monObjet.maMethode()) k; 

	template<
            class T,
            class U = decltype(T().begin()),
            class W = decltype(T().end())
        >
     .... 
\end{lstlisting}
\end{frame}

\begin{frame}[containsverbatim]
\lstinline{std::declval} : répond au problème suivant : Si on veut connaître le type de retour d'un méthode donc doit faire un decltype sur l'appel de la méthode. Mais comment faire lorsque le type n'est pas default constructible ? 
	\begin{lstlisting}
class W = decltype(T().end())// marche si T() est valide
class W = decltype(std::declval<T>().end()) // marche.
	\end{lstlisting}
	
\lstinline{std::enbale_if} : \lstinline{enable_if<false>::type} n'existe pas et \lstinline{std::enable_if<true>::type} existe. \lstinline{enbale_if} prend un deuxième argument optionnel qui donne le type de "\lstinline{::type}", par défaut il vaut void.
\\ 
\lstinline{std::is_same} : \lstinline{is_same<T,U>::value} vaut vrais si T==U et faux si non.

\end{frame}

\subsection{Ecrire un type\_trait}
\begin{frame}[containsverbatim]
\begin{itemize}
	\item Hériter de \lstinline{std::true_type} ou \lstinline{std::false_type} nous permet de déclarer rapidement un \lstinline{constexpr static bool value=true/false;}
	\item On a tout ce qui nous faut !
\end{itemize}


Exemple : \href{run:../code_demo/traits.cpp}{traits.cpp}\\

\end{frame}

\subsection{Remarques}
\begin{frame}[containsverbatim]
	Ce code ne marche pas : error: redefinition of template<class T, class U> void foo(const T\&)
	\begin{lstlisting}
template< class T,
          class U = typename std::enable_if
                    < is_container<T>::value>::type>
void foo (const T& t) {
    std::cout <<"je match les conteneurs"<< std::endl;
}
template< class T, 
          class U =  typename std::enable_if
                     < !is_container<T>::value>::type>
void foo (const T& t) {
    std::cout<<"je match pas les conteneurs"<<std::endl;
}
	\end{lstlisting}
\end{frame}
\begin{frame}[containsverbatim]
	Ce code marche, mais le passage à l’échelle est mauvais :/
	\begin{lstlisting}
template< class T,
          class U = typename std::enable_if
                    < is_container<T>::value>::type>
void foo (const T& t) {
    std::cout <<"je match les conteneurs"<< std::endl;
}
template< class T, 
          class U =  typename std::enable_if
                     < !is_container<T>::value>::type,
          class V = void >
void foo (const T& t) {
    std::cout<<"je match pas les conteneurs"<<std::endl;
}
	\end{lstlisting}
\end{frame}
\begin{frame}[containsverbatim]
	Celui la marche, passe à l’échelle, mais la lecture du type de retour est moins directe. 
	\begin{lstlisting}
template< class T >
typename std::enable_if< 
        is_container<T>::value, void  >::type
foo (const T& t) {
    std::cout <<"je match les conteneurs"<< std::endl;
}
template< class T >
typename std::enable_if< 
        ! is_container<T>::value , void  >::type  
foo (const T& t) {
    std::cout<<"je match pas les conteneurs"<<std::endl;
}
	\end{lstlisting}

\end{frame}
\begin{frame}[containsverbatim]
	On peut aussi utiliser un paramètre par défaut, en utilisant un pointeur que l'on initialise à null.
	\begin{lstlisting}
template < class T>
void foo (const T&, typename std::enable_if<
              is_container<T>::value>::type* = nullptr) {
   std::cout <<"je match les conteneurs"<< std::endl;
}

template < class T>
void foo (const T&, typename std::enable_if<
             !is_container<T>::value>::type* = nullptr) {
   std::cout <<"je match pas les conteneurs"<< std::endl;
}
	\end{lstlisting}

\end{frame}
\begin{frame}[containsverbatim]
Et enfin on peut utiliser du type dispatching :
	\begin{lstlisting}
template < class T>
void foo  (T&& t , std::true_type  )
{
    std::cout<<"je match les conteneurs"<<std::endl;
}

template < class T>
void foo  (T&& t , std::false_type   )
{
    std::cout<<"je match pas les conteneurs"<<std::endl;
}

template < class T>
void foo (T&& t) {
    foo(std::forward<T>(t), is_container<T> ()); 
}
	\end{lstlisting}

\end{frame}

\begin{frame}[containsverbatim]
	\frametitle{Fonction d'aide}
	On peut utiliser simplifier l’écriture du code utilisateur que ça soit au niveau du type de retour ou grâce à la déduction automatique du type.
	\begin{lstlisting}
decltype(typeRetour(lambda))
vs
decltype(getInformationParam(lambda))::type::
                                  result_type>::value
vs
type_result<decltype(lambda_test)> 
	\end{lstlisting}
Le dernier utilise \texttt{c++14} et les usings template, c'est sans doute le plus simple lorsqu'un utilise une classe "callable".
	\begin{lstlisting}
decltype(typeRetour<structParenthese>())
vs
type_result<structParenthese>  
	\end{lstlisting}

\end{frame}
\section{Déduction de type}
\begin{frame}[containsverbatim]
On stocke simplement les informations que l'on veut
\begin{lstlisting}
template<class ReturnType, class ... Args>
struct informationParamParamFactorisation{
    constexpr static size_t arity = sizeof...(Args);
    using  result_type = ReturnType;
    template <size_t indice>
    struct arg_type_{
       static_assert((indice < arity ), "msg d'erreur");
       using type= typename std::tuple_element<indice,
                           std::tuple<Args...>>::type;

    };
    template <size_t i> using arg_type = 
                           typename arg_type_<i>::type;
};
\end{lstlisting}
\end{frame}
\begin{frame}[containsverbatim]
On utilise le fait que les fonctions templates les plus spécialisées ont la priorité.
\begin{lstlisting}
template <class T> struct informationParam 

template <typename ClassType, typename ReturnType,
                                      typename... Args>
struct informationParam<ReturnType(ClassType::*)
                                             (Args...)>
template <typename ClassType, typename ReturnType,
                                      typename... Args>
struct informationParam<ReturnType(ClassType::*)
                                       (Args...) const>

template < typename ReturnType, typename... Args>
struct informationParam<ReturnType(*)(Args...)>
\end{lstlisting}

\end{frame} 


 

\begin{frame}[containsverbatim]
Maintenant on prend soin de nos utilisateurs en mettant des erreurs claires. 
\begin{lstlisting}
template <class... T>
class ERREUR;
template <class T>
struct informationParam{
    struct IsNotACallable {};
    ERREUR<T,IsNotACallable> erreur;
};
vs
template <class T>
struct informationParam{
    static_assert(
         std::is_same < T, typename void_if_valide<T>
                                ::type >::value &&false,
               "L'argument template n'est pas callable"
              );};

\end{lstlisting}
\end{frame}

\begin{frame}
Si on résume, ça nous donne : 
	\href{run:../code/src/paramInfo.hpp}{paramInfo.hpp}\\


\end{frame}

%\begin{frame}[containsverbatim]
%La fonction qui donne le nombre d'argument, j'ai pas réussi à faire un beau message d'erreur, du coup j'ai mis un commentaire la ou le compilateur va envoyer l'utilisateur. Mieux que rien...
%\begin{lstlisting}
%template<class T>
%constexpr size_t nbParam(T fonction){
%    // si ça plante ici c'est que ta classe elle a pas
%                                  d'operateur () , noob
%    return decltype(getInformationParam(fonction))
%                                         ::type::arity;
%}
%template<class T> constexpr size_t nbParam(){
%    return decltype(getInformationParam(std::declval<T>()))
%    			::type::arity; // idem (pas de place dans le slide)
%}
%\end{lstlisting}
%
%\end{frame} 

\begin{frame}[containsverbatim]
La fonction principale, avec deux écritures en fonction de ce qui est le plus simple pour l'utilisateur. 
\begin{lstlisting}
template<size_t nb,class T>
constexpr typename decltype(getInformationParam(
    std::declval<T>()))::type::template arg_type<nb>
    typeParam(T fonction );

template<size_t nb,class T>
constexpr typename decltype(getInformationParam(
     std::declval<T>()))::type::template arg_type<nb>
     typeParam();
\end{lstlisting}

\end{frame} 


\begin{frame}[containsverbatim]
Et puisque que ça me coûte rien je fais le même avec le type de retour.  
\begin{lstlisting}
template<class T>
constexpr typename decltype(getInformationParam(
    std::declval<T>()))::type::result_type 
    typeRetour(T fonction );

template<class T>
constexpr typename decltype(getInformationParam(
    std::declval<T>()))::type::result_type 
    typeRetour( );
\end{lstlisting}

\end{frame} 

\begin{frame}
Puisqu'on est chaud on peut regarde l'application au notre \lstinline{std::find_if} \\ source : \href{run:../code/src/main.cpp}{main.cpp}\\


\end{frame}

\begin{frame}
    \begin{center}
        \Huge Questions ? 
    \end{center}
\end{frame}



\end{document}
