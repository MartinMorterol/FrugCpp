\documentclass{beamer}

\usepackage[utf8]{inputenc}  
\usepackage[french]{babel}
\usepackage[T1]{fontenc} 
\usepackage{listings}


\usepackage{multimedia}
\usepackage{graphicx}
\usepackage[normalem]{ulem} %tentative de barer un text
\usepackage{tabularx}

\graphicspath{images/}
\usetheme{Luebeck}
\usecolortheme{crane}


% un test pour pouvoir barer qqch dans le titre de beamer
\newlength{\textlarg}
\newcommand{\barre}[1]{%
   \settowidth{\textlarg}{#1}
   #1\hspace{-\textlarg}\rule[0.5ex]{\textlarg}{0.5pt}}

% la gestion des sommaires
\AtBeginSection[]
{
 \begin{frame}
  \frametitle{Sommaire}
  \tableofcontents[currentsection, hideothersubsections]
  \end{frame} 
}


\AtBeginSubsection[]
{
 \begin{frame}
  \frametitle{Sommaire}
  \tableofcontents[  
    sectionstyle=show/shaded,
    subsectionstyle=show/shaded/hide
    ]
  \end{frame} 
}

% l'inbrication de itemize
\setbeamertemplate{itemize subitem}{\tiny\raise1.5pt\hbox{\donotcoloroutermaths$\blacktriangleright$}}

% le pied de page en 3 parties avec des n° de ligne
\setbeamertemplate{footline}{
\leavevmode%
\hbox{\hspace*{-0.06cm}
\begin{beamercolorbox}[wd=.2\paperwidth,ht=2.25ex,dp=1ex,center]{author in head/foot}%
	\usebeamerfont{author in head/foot}\insertshortauthor%~~(\insertshortinstitute)
\end{beamercolorbox}%
\begin{beamercolorbox}[wd=.6\paperwidth,ht=2.25ex,dp=1ex,center]{section in head/foot}%
	\usebeamerfont{section in head/foot}\insertshorttitle
\end{beamercolorbox}%
\begin{beamercolorbox}[wd=.2\paperwidth,ht=2.25ex,dp=1ex,right]{section in head/foot}%
	\usebeamerfont{section in head/foot}
	\insertframenumber{} / \inserttotalframenumber\hspace*{2ex}
\end{beamercolorbox}}%
\vskip0pt%
}




\title{Déduction des types des paramètres}
\author{\url{Martin Morterol}}
\date{\today}


\begin{document}

%le titre
\begin{frame}[plain]
\titlepage
\end{frame}

\section{Pourquoi ?}

\begin{frame}
	Contexte : Un petit jeu en 2D.
    \begin{itemize}
    	\item Une carte est un conteneur au sens de la STL.
    	\item On peut donc l'utiliser dans un for range.
    \end{itemize}
    Problème : 
    \begin{itemize}
    	\item Une carte contient des std::reference\_wrappeur<Case>.
    	\item J'ai pas envie que l'utilisateur connaisse ma structure interne ni qu'il en dépende.
    	\item J'ai pas envie d’écrire .get() à chaque fois. 
    	\item J'ai envie de me faire plaisir ça sert à ça les projets perso :)
    \end{itemize}
    Solution : Je crée mon iterateur maison que j'appelle FoncteurIterator.
    
\end{frame}

\begin{frame}
	\href{run:FoncteurIterator.cpp}{FoncteurIterator.cpp}\\
	Problème : 
	\begin{itemize}
		\item Pour utiliser un std::function<> J'ai besoin du type de retour du foncteur et du type du paramètre. 
		\item J'ai également besoin du type du paramètre pour d'autre chose (l'operator::* par exemple)
	\end{itemize}
	Solution :
	\begin{itemize}
		\item FoncteurIterator< IteratorTemplates, FoncteurTemplate, typeRetour, typeParam>, Ok mais j'aime pas 
		\item Trouver un moyen de déduire typeRetour et typeParam directement de FoncteurTemplate. 
	\end{itemize}
\end{frame}

\section{Comment ?}


\begin{frame}[containsverbatim]
On utilise le fait que les fonctions templates les plus spécialisées ont la priorité.
\begin{verbatim}
template <class T> struct informationParam 

template <typename ClassType, typename ReturnType,
          typename... Args>
struct informationParam<ReturnType(ClassType::*)(Args...) >

template <typename ClassType, typename ReturnType,
          typename... Args>
struct informationParam<ReturnType(ClassType::*)(Args...)
                        const>

template < typename ReturnType, typename... Args>
struct informationParam<ReturnType(*)(Args...)>
\end{verbatim}

\end{frame} 


\begin{frame}[containsverbatim]
Le gros de la magie est fait, il reste à faire le traitement.
\begin{verbatim}
template<class ReturnType, class ... Args>
struct informationParamParamFactorisation{
    constexpr static size_t arity = sizeof...(Args);
    using  result_type = ReturnType;
    template <size_t indice>
    struct arg_type_{
        static_assert ((indice < arity ), "msg d'erreur" );
        using type= typename std::tuple_element<indice,
                    std::tuple<Args...>>::type;

    };
    template <size_t i> using arg_type = 
                               typename arg_type_<i>::type;
};
\end{verbatim}
\end{frame} 

\begin{frame}[containsverbatim]
Maintenant on prend soin de nos utilisateurs en mettant des erreurs claires. 
\begin{verbatim}
template <class... T>
class ERREUR;

// Le cas general si on arrive la c'est qu'on a perdu,
//  on compile pas et on informe l'utilisateur
//  que "ERREUR<LeParamNestPasUneFonction>"

template <class T>
struct informationParam{
    struct LeParamNestPasUneFonction {};
    ERREUR<LeParamNestPasUneFonction> erreur;
};
\end{verbatim}
\end{frame}


\begin{frame}
Si on résume, ça nous donne : 
	\href{run:paramInfo.hpp}{paramInfo.hpp}\\


\end{frame}


\section{L'utilisation}
\begin{frame}[containsverbatim]
La fonction qui donne le nombre d'argument, j'ai pas réussi à faire un beau message d'erreur, du coup j'ai mis un commentaire la ou le compilateur va envoyer l'utilisateur. Mieux que rien...
\begin{verbatim}
template<class T>
constexpr size_t nbParam(T fonction){
    // si ça plante ici c'est que ta classe elle a pas
                                  d'operateur () , noob
    return decltype(getInformationParam(fonction))
                                         ::type::arity;
}
template<class T> constexpr size_t nbParam(){
    return decltype(getInformationParam(std::declval<T>()))
    			::type::arity; // idem (pas de place dans le slide)
}
\end{verbatim}

\end{frame} 

\begin{frame}[containsverbatim]
La fonction principale, avec deux écritures en fonction de ce qui est le plus simple pour l'utilisateur. 
\begin{verbatim}
template<size_t nb,class T>
constexpr typename decltype(getInformationParam(
    std::declval<T>()))::type::template arg_type<nb>
    typeParam(T fonction );

template<size_t nb,class T>
constexpr typename decltype(getInformationParam(
     std::declval<T>()))::type::template arg_type<nb>
     typeParam();
\end{verbatim}

\end{frame} 


\begin{frame}[containsverbatim]
Et puisque que ça me coûte rien je fais le même avec le type de retour.  
\begin{verbatim}
template<class T>
constexpr typename decltype(getInformationParam(
    std::declval<T>()))::type::result_type 
    typeRetour(T fonction );

template<class T>
constexpr typename decltype(getInformationParam(
    std::declval<T>()))::type::result_type 
    typeRetour( );
\end{verbatim}

\end{frame} 



\begin{frame}
    \begin{center}
        \Huge Questions ? 
    \end{center}
\end{frame}



\end{document}
